\chapter{Discussion}
\lhead{\emph{Discussion}}

\section{Scientific results}

Starting out with recognizing symbols using a convolutional neural network was a decision made early in the project. A lot of previous research on recognizing similar images such as with the MNIST dataset already existed. Therefore, using images as input data was a safe direction when creating a minimal viable product (MVP).


\subsection{Neural network accuracy}
The convolutional neural network's performance can be seen in figure \ref{fig:CNN_44000}. Its performance on the test dataset was very good, and it kept increasing over time. However, as a consequence of our relatively small train dataset, the network did not show the same accuracy on the validation data and the real dataset. This model was prone to overfitting, even when including several dropout layers. 

When traces are converted to images, a lot of sequential data is lost. Similar looking symbols are often drawn in different ways, however end up looking similar when the drawing is finished. This is information lost when turning the input data into images, and therefore may result in lower prediction accuracy than a network which includes this information.

A recurrent neural network was used in order to include the sequential information in our data. Creating a RNN model introduces its own challenges, especially the format on its input data. A trace group may include data with different number of coordinates and different number of traces. A lot of work was therefore spent on turning the input data into a format the RNN model could predict. Downsampling to reduce the number of points, padding the data arrays, combining traces to a single trace all while still keeping information about which traces were separate. A lot of inspiration was taken from Google's Quick Draw, and its open source code. \cite{_recurrent_2017} 

As figure \ref{fig:RNN_44000} displays, the recurrent neural network managed to perform better on the validation dataset, and did not have the same problems with overfitting as the CNN model. The recurrent network had issues with underfitting the train dataset, even after 40 epochs. 

The two different models were very different, both in model architecture, and the format of input data. Therefore, it was interesting to see whether the models could perform better predictions combined, than each model individually. As seen in figure \ref{fig:combined_CNN_RNN_44000}, the combined model performed better than both the CNN and RNN on both the test dataset and the real dataset. 

With an accuracy of 98.28\% on the test dataset, the combined model had an accuracy 1.77\% better than the CNN model, and an accuracy of 0.93\% better than the RNN. The difference was even larger on the real dataset, with 1.52\% better accuracy than the RNN. This indicates that the combined model handles data from different sources better. The 'real' dataset is small compared to the other datasets, so it is hard to make conclusions on the 'real' dataset. 

This indicates that a combined artificial neural network could in many cases outperform a CNN or RNN with the similar datasets on this specific classification problem. 

\subsection{Sources of error}
% skrivemåte
% overfitting, underfitting, vanishing, exploding, dead neurons (learning rate)
% preprocessing

Using images with low resolution and only a single value for the alpha channel (either black or white) is a possible source of error in our solution. Information about edges in the symbols is lost, as the position of a trace will be rounded to the nearest pixel value. An attempted solution was to generate pixels with a lower alpha value in the surrounding pixels. This could have counteracted the negative effects of rounding, however it did not result in a better accuracy, perhaps because of bugs in the preprocessing.

Preprocessing has a large impact on how good the model performs. The preprocessing done in this project was based on research papers, examples, as well as some intuition. This intuition is often backed by comparing results for different preprocessing methods. A change in preprocessing technique resulted in regeneration of training data, and new training of the models. As the generation and training is CPU intensive, finding the optimal solutions were restricted by time and resources. Therefore, it is difficult to reason that the chosen techniques were the optimal techniques. An example of this intuition is when choosing the best range to scale the traces, with image size of 26 pixels, and scaling the traces in the range $[-1, 1]$. 

When comparing the accuracy for the three different models, many sources of errors have the possibility to skew the results. Without knowing the optimal RNN model and CNN model, determining whether a combination of both performs better than each individually is difficult. The combined model included its own fully connected layer, which increases the depth of the model. In research, deeper models have often increased prediction accuracy, and this layer can therefore be the reason for the combined model's improved accuracy.

As with testing hypothesis in preprocessing techniques, training the models were time intensive tasks. Deciding the best architectures and model parameters was therefore also restricted by time and resources, and the final models are consequently most likely far from the optimal models.

% Et avsnitt om størrelse på dataset

\section{Engineering results}
\label{engineering_results}

%Hvordan ble sluttproduktet?
%Fikk oppdragsgiver det som var forventet? 
%Hvilke krav ble oppfylt? 
%Hvilke krav ble ikke oppfylt?

Since the product derived from a research project, there was never any requirements set. However, the goal was to have an application supporting the functionality of Matistikk, and to have a classification rate needed for it to be an asset. A lower bound of accuracy was not agreed upon because it was difficult to set realistic goals not knowing the potential of the project. 

Segmentation turned out to be the most complex and difficult issue to solve in this project, this is why it's included in engineering results. The focus in this project was to create both a working application and a solid classifier. Making a general solution for segmentation of symbols was attempted early in the project. The solution consisted of making a CNN model generate bounding boxes around symbols. After bounding boxes were generated, we could extract the traces inside the bounding box and combine them to a segmented trace group. Because of poor results in generating bounding boxes as a consequence of a small dataset and an unfit model, we abandoned this solution.

Using a recurrent neural network to determine the symbols' context was proposed, however there was not enough time to attempt this approach. The context determination could be done by sending predicted bounding boxes and the classified symbols to a RNN, which would thereby generate a resulting latex expression. 

It was clear that some sort of machine learning solution on the segmentation and context issues would be preferable. In the case of segmenting the symbols, it was difficult to create a solution that got better results than an algorithm recognizing overlapping traces. In context search, limited time and experience with machine learning caused us to choose an algorithmic solution to find context.

%Hvorfor ble resultatene som de ble?
The engineering results of this project was dependent on the fact that we had access to training data from the CROHME dataset. A large part of the project was reserved for creating training data, but this was now time we could spend on developing a good model. The access to the CROHME data influenced our architectural decisions as well. In order to create a working system for our product owners, it was necessary that data from the end-user eventually was on the same format as the training data.

As machine learning was a new concept for all the team members, some of the decisions taken focused on other solutions than machine learning. We decided to do segmentation and context search through traditional methods, a decision which was influenced by the lack of experience using machine learning techniques. Using traditional methods restricted the projects generalization, and means expanding the software's functionality involves more work.

\subsection{Strengths and weaknesses}

The system developed has both strengths and weaknesses. There are details in all parts of the system that could have been better, however, many of these could be solved if the projects time scope lasted longer. Overall the system does a great job if the input data contains small amounts of noise and is able to segment the symbols correctly. % si noe om mengder detalj den kanskje ikke klarer?

% Traing data
In the training phase of the neural network there were some limitations in terms of training data. The training set was small and many of the symbols used did not have enough training samples to train the neural network optimally. More training samples could have been collected, a system for this would require some time to set up. However, the model trained handles most symbols well and achieves high accuracy for the validation sets.

% Segmentation
Since the segmentation step in this approach is very simple and occurs early in the recognition process. This might lead to many errors in the following steps, especially the classification. If a overlap between two symbols is mistakenly drawn, the classification will most likely fail, as the traces for each symbol will be combined when classified. Another issue with the segmentation is that an overlap might not be detected. In some cases, the distances between the points in a trace is relatively large, and if another trace crosses in this region an overlap could go unnoticed. The segmentation system does doubles the amount of points for all traces before overlap is checked, however, this is still not enough in some cases.

% Interpretation
Finding the context of the input data is accomplished with a recursive search and a set of hard coded rules. This will ensure consistency for similar input data, but is limited in terms of understanding mathematical notation. Each notation to support needs its own set of rules, and the complexity of the context search would increase with each of them. Complex structures such as integrals and sums is not supported since they require new methods to search for symbols within their bounds. In this project it was chosen to not focus on this, due to time restrictions.

Some of the rules in the context search uses static values instead of dynamic values, such as the size of a symbol. To find the multiplication sign dots the interpretation system only checks if a symbols is smaller than a threshold. This is not a general solution, as some persons might write larger multiplication dots than others. A better solution would be to calculate the static values and thresholds based on the size and propositions of all the input symbols, however, this is due for further work.

% Front end 
The front end demonstration developed in this project has some issues. Especially during erasing of traces, and drawing on the same page over a long time. The demonstration was not regarded as a priority, and will therefore need some bug-fixing when integrating to Matistikk. On the other hand, its core functionality of drawing and sending the drawings has been thoroughly tested, and should therefore work as expected.

\subsection{Compared to other solutions}
%Sammenlign med andre løsninger: myscript, drawpix math, forskningsartikler

MyScript is a commercial solution that has participated and won the CROHME competitions. Comparing our solution to theirs is difficult, since they used a private dataset to train their models. However, there exists a demo site which displays its features and support for many different symbols. MyScript has superior classification abilities to the solution made in this project. Khan Academy, a popular learning platform, recently deployed a new app for iPads, which uses MyScript to recognize user input.

Another solution offering similar functionality is Mathpix \cite{mathpix}, a tool for converting images to Latex. This software is also proprietary. Mathpix utilises only images for the recognition, however, their solution still offers more functionality than ours. This includes a larger symbol bank as well as being able to recognize more complex symbol combinations.

\section{Administrative results}

From the project participants point of view, the project progress had ups and downs. During work with machine learning, progress often stagnated, which affected the motivation. Following principles from modern agile and pair programming, we worked together to incrementally make progress. These principles were therefore regarded as an important factor in our progression.

From the aspect of technology, we were very pleased with using Python, Keras and Tensorflow. These were technologies which hid much of the complexity with machine learning. Especially Keras worked well with the agile principles, through quick prototyping and the ability to create an early MVP. Tensorflow as a Keras back end enabled us to do the computations on a GPU, which increased our ability experimentation with different models.

%Hva ble ikke bra på grunn av valgt prosess, fremgangsmåte og teknologi?
Using modern agile with inspiration from lean startup worked well in the sprints dominated by software development. Phases with mostly machine learning did not benefit in the same way because it is difficult to plan and estimate progression dealing with complex new subjects and approaches. However, small incremental steps were inspired by these methods, and this helped us make continuous progress.

%Hva ble bra eller dårlig uavhengig av valgt prosess, fremgangsmåte og teknologi?
Independent of the choices made on process, approach and technology the teamwork worked well. Communication through Slack worked as expected, frequent status meetings and thorough discussions led to good results in both software and classification accuracy.

\section{Impact}

%Dere skal også drøfte arbeidet i forhold til et helhetlig systemperspektiv.

%Sett resultatene inn i en samfunnsmessig og økonomisk, eventuelt også miljømessig, sammenheng.
% Hvordan påvirker applikasjonen brukerne?
% 

The product created is meant to be a resource for Matistikk. It can function as both stand-alone and as a part of a larger system. The addition of our application increases Matistikk's potential to be a substitute for traditional pen and paper in mathematical education.

As the data used is this product is based on data distributed for academic use, there are some restrictions to its usage. The model generated from the data may therefore not be used in a commercial setting, and in order for Matistikk to be proprietary software, it would need its own dataset.

The technical proficiencies needed to use this product is average. Either drawing has to be done through a touch screen, iPad or similar, which is technologies most people has experience with. This product will be an addition to an already existing application Matistikk, where the target audience are education students. Education students are expected to have a technical proficiency sufficient to handle this web application.

\subsection{Ethics}
%however, some modifications will have to be done to use the product in a commercial setting

The project does not limit where and how Matistikk in a non-commercial setting. This indicates that teachers with limited technical capabilities may have to use our product. If the technical proficiency is insufficient, this can have a negative impact on the quality of education these teacher can provide. 

The product created gives the opportunity to reduce people's handwriting to a general Latex format. This increases the objectivity in test evaluation, as people with different handwriting styles are evaluated equally. On the other hand, a misclassification will give teachers less opportunity for subjectively evaluating students, in cases where this would be advantageous.

\subsection{Professional ethics}

When developing the product, a lot of attention was directed to writing good maintainable code. As this is code that may be worked on by future developers, we wished to give them the best opportunity possible to quickly learn and use our product. Giving credit for code not written by us was also important, and all code taken from other people has been documented. 

Relative to our work as a team. We wished to make each other's workday as good as possible. All the team members had part time work, and we therefore reserved one day a week for people to work on this. We tried to listen to each other's ideas, and help each other perform better.

In addition to having good internal teamwork we also communicated, respected and included our product owners where we felt it was suitable. Sharing progress was an important aspect of our project flow. Using email and setting up meetings in order to make our product owners included and up to date was therefore a priority.

\section{Teamwork}
%Studenter som jobber i gruppe skal skrive et avsnittder de reflekterer over gruppearbeidet og hvordan det har fungert.(Dette kommer i tillegg til det individuelle refleksjonsnotatet som skal leveres.
A quick reflection of the teamwork and team dynamics during this project reveals curious students eager to learn in an exciting field. It is though, not hard to conclude that in a difficult subject experiencing both productivity and motivational ups and downs is common. During the project, Thursdays were often reserved for part-time jobs. Luckily everyone could work on Thursdays, so no other weekday went away to inefficient teamwork. Every team member was responsible for their own part of the work and to get the recommended amount of hours, which was about 30 hours per week. In retrospect, time working together is crucial so it is important to stay focused on the project and not work, although this has not been a limiting factor.