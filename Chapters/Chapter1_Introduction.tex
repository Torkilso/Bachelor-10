\chapter{Introduction}
\lhead{\emph{Introduction}}

\section{Motivation}

Usage of digital tools in education is increasing, and testing student's performance is moving toward digital platforms. However, mathematics is a subject where most testing is still done on pen and paper. Few good digital tools exist for writing and validating mathematical handwritten expressions. By using machine learning to validate student's submission, the benefits of using digital tools for tests within the subject may improve, and therefore increase adoption of digital tools for mathematics.

\section{Problem definition} 
\label{problem_definition}

Choosing a scientific problem or hypothesis turned out to be a challenge not knowing the potential of the assignment. Learning and experiencing from previous work on the same subject led us to explore convolutional and recurrent networks. After experimentation with both recurrent- and convolutional networks, we saw a potential use case for a combination of both types of models. Therefore, the following hypothesis emerged.

How does a combination of recurrent and convolutional neural network's accuracy perform towards classifying handwritten mathematical symbols?

\section{Structure of the report}
\subsection{Chapter 1 Introduction}
The introduction consists of the motivation behind the project, the problem definition and a list of abbreviations.

\subsection{Chapter 2 Theory}
To understand this thesis and choices made, a lot of theory has to be understood. This chapter consists of various topics in handwriting recognition and machine learning that are relevant for the assignment.

\subsection{Chapter 3 Technology and Method}
This chapter will specify the technology and how it is applied into our product. The purpose of this chapter is to give insight to how the results can be reproduced.

\subsection{Chapter 4 Results}
\begin{enumerate}
    \item \textbf{Scientific results}
    
    This section will describe the results associated with the problem definition in section \ref{problem_definition}.
    
    \item \textbf{Engineering results}
    
    This section will describe the status of the system at the time of delivery and how the system fulfils the assignment.
    
    \item \textbf{Administrative results}
    
    This section will describe the goals in terms of project planning and time management, as well as how these goals changed accordingly to progress and methodology used.
\end{enumerate}

\subsection{Chapter 5 Discussion}
This chapter consist of discussion about the results and how they were achieved.
\begin{enumerate}
    \item \textbf{Scientific results}
    
    A discussion around our the hypothesis from the results presented, with focus on possible errors and areas of improvement.
    
    \item \textbf{Engineering results}
    
    This section will discuss the software created in this project, including strengths and weaknesses. In addition, the system is compared to other solutions.
    
    \item \textbf{Administrative results}
    
    Describes what went good or bad as a result of choosing a specific process, approach or technology. Including what went good or bad regardless of the chosen process, method or technology.
    
    \item \textbf{Impact}
    
    Describes the impact of the system created.
    
    \item \textbf{Teamwork}
    
    A discussion about the participants' teamwork during the project.
    
\end{enumerate}
\subsection{Chapter 6 Conclusion}
This chapter consists of our conclusion regarding our scientific problem and comments on further work.

\section{Target audience}
Machine learning is an extremely large field, with enormous potential. We encourage the reader to be open to some technical aspects including thoughts and choices made during the project. Machine learning is quite complex and we hope to inspire students and others new to machine learning. With that said, both our report and code are meant for readers with some knowledge about software engineering, linear algebra and numerical mathematics. We also expect the reader to have a basic understanding about web applications.

\section{Purpose}
The purpose of this project is both to create something useful for our product owners, while exploring the possibilities and limitations of recognition of handwritten mathematical symbols. In addition, we hope to inspire further work on the subject. 

\section{Abbreviations}
\begin{table}[H]
\begin{tabular}{ l l }
\textbf{ANN} & Artificial Neural Network \\
\textbf{API} & Application Programming Interface \\
\textbf{BMP} & Bitmap \\
\textbf{CNN} & Convolutional Neural Network \\
\textbf{CPU} & Central Processing Unit \\
\textbf{CROHME} & Competition on Recognition of Online Handwritten Mathematical Expressions\\
\textbf{CSS} & Cascading Style Sheets \\
\textbf{GPU} & Graphics Processing Unit \\
\textbf{GRU} & Gated Recurrent Unit \\
\textbf{GUI} & Graphical User Interface \\
\textbf{HTML} & HyperText Markup Language \\
\textbf{HTTP} & HyperText Transfer Protocol \\
\textbf{ICFHR} & International Conference on Frontiers in Handwriting Recognition \\
\textbf{IDE} & Integrated Development Environment \\
\textbf{ILU} & Department of Teacher Education \\
\textbf{InkML} & Ink Markup Language \\
\textbf{JSON} & JavaScript Object Notation \\
\textbf{LSTM} & Long Short-Term Memory \\
\textbf{MLP} & Multilayer Perceptrons \\
\textbf{MNIST} & Modified National Institute of Standards and Technology database \\
\textbf{NTNU} & Norwegian University of Science and Technology \\
\textbf{OCR} & Optical Character Recognition \\
\textbf{RDP} & Ramer-Douglas-Peucker algorithm \\
\textbf{ReLU} & Rectified Linear Unit \\
\textbf{RNN} & Recurrent Neural Network \\

\end{tabular}
\label{table:abbreviations}
\caption{A list of abbreviations used, in alphabetical order.}
\end{table}