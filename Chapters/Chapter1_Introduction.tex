\chapter{Introduction}
\lhead{\emph{Introduction}}  % Set the left side page header to "List if Figures"

\section{Motivation} % needs restructuring, we are finding out a way to combine cnn and rnn's, not develop 
% proof of concept++++++

In education, the usage of digital tools has increased. 

%static input system
%preferable with more dynamic input
%more freedom to the student
%the student can show their thoughts in a more precise way (?)
%Something about matistikk

%Some of these tools uses a static system to let the students answer tasks, such as Matistikk, which is a web application for practising math. The main


In education the usage of digital tools to present and distribute tasks to students has increased [0]. % trenger kilde

This includes tasks with multiple choice, graphs, geometric models and algebra. Traditionally these tools does not support dynamic input in form of digital ink, which makes the tools somewhat unrealistic from classic math on paper.
%To compensate for a static 

% this needs data to backup my statements
% FUCK JEG TRENGER FAKTA! %Matistikk does have Geogebra integration which is working as intended, but a model for interpreting handwritten math is desired because it would, if implemented in a right way increase the efficiency for both the teachers and the students.
\section{Problem definition} % TODO review if this is actually somehting that we can find out etc...% 1st draft
How can proportionality and relativity be used in recurrent neural networks? \\ Our problem was created after being presented with our project task, which is a task to create a module for interpreting handwritten mathematical symbols. The task itself was focused on creating an generic module which could easily be used in both existing and upcoming projects for the department of teacher education at Norwegian University of Science and Technology.
%Eventually the project came to a crossroad between creating a module which were to serve 


\section{Structure of the report}
\subsection{Chapter 1 Introduction}
The introduction introduces important thoughts, concepts and motivation behind this project without going in depth.

All of the concepts introduced in this chapter are required to understand both relevance and motivation.

\subsection{Chapter 2 Theory}
The theory consists of relevant theory to understand both approach, limitations, choices and eventually conclusion.\\
By our project definition we are to expect that the reader is at a level of a 2nd year computer engineering student at our own department. \\
Even though we know that the 2nd year computer engineering student does have the conceptual knowledge required to understand the theory behind neural networks, some of the mathematics will be explained.

\subsection{Chapter 3 Method}
This chapter has a direct correlation with chapter two. Here the choices and technology in chapter two are justified. 
In addition, we will also discuss tools that made our project flow optimal.

% IS THIS RIGHT? % "Referansearkitektur????"
% Some algorithms, general domain model......

\subsection{Chapter 4 Results}
The results are divided into at least three parts.
\begin{enumerate}
    \item Scientific results
    This section will describe the data/product which will serve as a foundation for answering our problem/hypothesis in chapter 1.1.
    
    \item Engineering results
    This section will describe the goals set in the start of this project. In a clearly defined product and goal this would be the vision document, in this projects case it would be much more up to the product owners. %??
    
    \item Administrative results
    This section will go into the process behind this project, from the planning stage to the finishing touches on the thesis. In addition to the process we will also talk about our goals set early in the project phase, and how our goals changed according to our progress and knowledge. Eventually we will arrive at methodology and how there was made attempts to follow agile methodologies.
    
\end{enumerate}

\subsection{Chapter 5 Discussion}
The discussion is also divided into the same parts as Chapter 4. % TODO eholde hvem?
In this chapter we are to answer some questions.
%Why, or why didn't or hypothesis hold?

To explain the different sections, the questions linked to each section is listed underneath.
\begin{enumerate}
    \item Scientific results
    Why did, or why didn't our hypothesis or scientific problem hold?
    In addition to discussing the hypothesis we are to discuss how our results can be interpreted in combination with or as an answer to our hypothesis or scientific problem.
    
    \item Engineering results
    How did the software part of this project turn out? Were the product owners satisfied with the result? Which requirements were met and which were not? Why did the results become what they are? What went good? What went bad?
    
    \item Administrative results
    What went good or bad as a result of choosing a specific process, approach or technology? What went good or bad regardless of the chosen process, method or technology?
    
\end{enumerate}

% TODO Svakheter => samfunn => etikk => miljø ...
% avsnitt som reflekterer over gruppearbeidet
% 
In addition to discussing our results, the weaknesses of this work should is to be discussed, how it could have been done differently and so forth. 

\subsection{Chapter 6 Conclusion}
The purpose of this work is to find ways of how to use machine learning to recognize handwritten mathematical symbols. Our product owners have a need, and our goal is to fulfill that need in the best way possible. We hope to accomplish exactly that by examining the different approaches to an issue like this and try to achieve desired results.\\
In addition to the project 

\subsection{Attachments}
This explains itself, our attachments are included here. 

\section{Target audience}
% INTRODUSERE TARGET AUDIENCE
Machine learning is an extremely large field, with enormous potential. We encourage the reader to be open to some technical aspects including our thoughts and process from start to finish. Machine learning is quite complex and we hope to inspire students and others new to machine learning. With that said, both our report and code is meant for a reader with some knowledge about software, linear algebra and numerical mathematics.
%Machine learning is a complicated field with a lot of attention these days, an article from The Guardian \cite{gibbs_googles_2018} states that Google's AI is beeing used in a drone project.

\subsection{Report} % Er dette et godt ord??
This project report is meant for everyone curious about an specific introduction to practical examples of machine learning using neural networks. The specific part is linked to this projects assignment, which is to use different approaches to achieve an respectable accuracy.\\
% line break
Even though this report is meant as an introduction to a complicated field, we have based our writing on that the reader has knowledge equivalent to an 2nd year computer engineering student at Norwegian University of Science and Technology in Trondheim. Thus, some of the fundamentals are excluded. % ok setning?


\subsection{Application}
The application which consists of an simple back end and a front end library is available to use under the MIT license \parencite{_mit_????}. As previously stated, the application is specifically created to fit the needs of our product owners, but the front end library can be used in a more general matter. % TODO bruk bilder, figurer eller lignende av applikasjonen (?)


%\section{CROHME} % Teori
%CROHME is an abbreviation for Competition on Recognition of Online Handwritten Mathematical Expressions.
%CROHME is a competition for recognizing online handwritten mathematical symbols. CROHME is organized by International Conference on Frontiers in Handwriting Recognition. \\ 
%CROHME 2016 \parencite{mouchere_icfhr2016_2016} concluded with that handwriting is still a challenge to be solved, even after six years of competitions. Ratings on the individual tasks was accomplished, but overall it requires first of all, the segmentation to be perfect. An error in the segmentation process will supply the next steps with incorrect information and a bad foundation to perform classification on.
% si noe om "junk" ? filtrering osv er ikke lett
%The winner of CROHME 2016 and other competition years, was MyScript. MyScript has a commercial solution which has rich functionality and good results, but they managed to achieve those results with their own private data set. \parencite{mouchere_icfhr2016_2016}  \\ 

\section{Purpose}
% Interpret mathematical symbols and expressions and figure out how rnn/cnn/... can be used to obtain desirable accuracy

%\section{Cooperation with the department of Teacher Education, NTNU}
% Owners of matistikk...
% Collection of data (?) and the help we needed to exclude/include some symbols.
The purpose of this project is both to create something useful for our product owners, while exploring the possibilities and limitations of recognition of handwritten mathematical symbols. In addition to creating something useful, we hope to inspire further work on the subject. Choices made during the project were strongly influenced by previous work, for example \cite{thoma_-line_2015} and \cite{lu_recognition_????}.
\section{Abbreviations}  % Set the left side page header to "Abbreviations"
\begin{table}[H]
\begin{tabular}{ l l }
\textbf{ANN} & Artificial Neural Network \\
\textbf{BMP} & Bitmap \\
\textbf{CNN} & Convolutional Neural Network \\
\textbf{CPU} & Central Processing Unit \\
\textbf{CROHME} & Competition on Recognition of Online Handwritten Mathematical Expressions\\
\textbf{GPU} & Graphics Processing Unit \\
\textbf{GUI} & Graphical User Interface \\
\textbf{ICFHR} & International Conference on Frontiers in Handwriting Recognition \\
\textbf{ILU} & Department of Teacher Education \\
\textbf{InkML} & Ink Markup Language \\
\textbf{JSON} & JavaScript Object Notation \\
\textbf{LSTM} & Long Short-Term Memory \\
\textbf{MLP} & Multilayer Perceptrons \\
\textbf{MNIST} & Modified National Institute of Standards and Technology database \\
\textbf{OCR} & Optical Character Recognition \\
\textbf{RNN} & Recurrent Neural Network \\
\end{tabular}
\label{table:abbreviations}
\caption{A list of abbreviations used.}
\end{table}
